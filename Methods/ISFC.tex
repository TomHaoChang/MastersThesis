\documentclass[12pt]{article}
\usepackage[utf8]{inputenc}
\usepackage{tikz}
\usepackage{amsmath,amsfonts,amsthm}
\usepackage[vlined, ruled]{algorithm2e}
\usepackage{geometry}
\usepackage[noend]{algpseudocode}
\usetikzlibrary{bayesnet}
\usepackage[nottoc,numbib]{tocbibind}
\setlength{\parskip}{1em}
\geometry{letterpaper,left=1.5in,right=1in,top=1in,bottom=1in}
\setlength\parindent{0pt}
\linespread{1}
\newcommand{\E}{\mathrm{E}}
\newcommand{\Var}{\mathrm{Var}}
\newcommand{\N}{\mathcal{N}}
\newcommand{\tr}{tr}

\begin{document}
\subsection{Introduction}
Human beings have an insatiable desire to understand, not only the things around us, but also ourselves. Why do we behave and respond the way we do \cite{hasson2012}? How do we learn and adapt \cite{hasson2004}\cite{hasson2005}? What do we believe or value \cite{Greene01}? These questions have mesmerized wise men for thousands of years, and the answers continue to evade us. For many of us, the key to all the secrets about ourselves is our brain, the central hub of all our thoughts and decision making. So, when a new technology emerged allowing us to study the brain in a quantitative manner, it opened countless doors of opportunities for those who possessed an aptitude for quantitative analysis and a hunger to learn more about our own identity.\\

Functional Magnetic Resonance Imaging, often referred to as fMRI, was first discovered and applied as a brain mapping method in 1990 by Seiji Ogawa \cite{Ogawa90}. Since then, this technology quickly popularized among Brain Science related research due to its unprecedented low health risk to subjects and its ability to accurately translate brain activities into highly-structured data \cite{Logothetis01}. Once the the highly abstract brain activities are converted into signs and digits, decoding patterns in the brain becomes a possibility. By treating each fMRI brain image as a feature vector, machine learning algorithms trained on a subset of the images may be used to distinguish cognitive information (e.g. which part of a movie someone is watching) from the held-out images \cite{Norman06}\cite{peterson12}\cite{peterson17}. These approaches typically treat each brain image in isolation, and attempt to identify patterns of activity associated with each of several candidate brain states. However, measurements from fMRI are often corrupted by noises created by the environment, human error, random neural activity, etc \cite{peterson11}. To address this issue, new branches of research has been emerging that specifically focuses on the extraction of useful information from noisy brain fMRI data by treating the brain images as a dynamic time series, and one of the most fruitful branches is functional connectivity \cite{peterson9} \cite{peterson19} \cite{peterson20}.\\

Functional connectivity analyses entail computing the correlations between the time series of activations each pair of brain regions exhibits. When we observe the brain, the neural activities can appear incredibly complex. But mechanics of brain are far from random: rather, the dynamic patterns of activity our brains exhibit are highly structured. Presumably this mirrors the complex but highly structured nature of our internal thoughts and experiences. Recently, it has become clear that important cognitive information is contained in higher-order brain patterns, such as the dynamic correlational structure of the data \cite{davidson2016}. When applying functional connectivity analyses on one subject in isolation, one can examine how (or whether) the correlational structure of these activity patterns (across a given set of brain regions) varies according to the cognitive task an experimental subject is performing \cite{Turke13}\cite{Rubinov2010}{peterson10}. Alternatively, the inter-subject functional correlation (ISFC) analyses applies functionally connectivity analyses across multiple subjects to isolate stimulus-dependent inter-regional correlations from intrinsic neural processes and non-neuronal noise \cite{hasson2016}\cite{jeremy2017}.\\

Over the past decade, functional connectivity analyses of brain data has evolved significantly \cite{olaf2005}\cite{khambhati2017}. However, there are three fundamental limitations to past approaches:\\
\begin{enumerate}
\item	\textbf{Correlation matrices are not scalable.} Examining pairwise correlation in brain data
produces a correlation matrix with O(n2) entries (where n is the number of brain regions). When n is large (e.g. the number of voxels in an fMRI volume), the full correlation matrix can become unwieldy to compute with \cite{Rubinov2010}\cite{Betzel2017}\cite{Craddock2012}\cite{Yeo2011}. Furthermore, if one wishes to examine higher-order patterns (e.g. how correlations between correlations change over time), the storage requirements of the resulting patterns increase exponentially.

\item \textbf{The sliding window approach is not well suited to studying dynamic activity.} Most approaches to calculating functional connectivity uses the sliding window method, where a window of set time length is selected to calculate the functional connectivity at one time point before the window is shift forward for following calculations \cite{enrico2011}\cite{elena2012}. One disadvantage of the sliding window approach is the loss of information for a number of time points equal to half of the sliding window length. Although many practical applications use buffers to make up for the loss, repeated application of the sliding window approach on the same dataset—--e.g. calculating the correlation of the correlation between nodes—--is impractical. In addition, the sliding window approach provides only a poor approximation of the moment-by-moment patterns at the heart of these representations.\\

\item \textbf{Only using node activations and first level dynamic patterns may not be enough.} The brain is a network of seemingly autonomous nodes that are actually intricately interconnected. Useful information about the brain may exist within interactions between interactions between nodes (2nd order), interactions between interactions between interactions (3rd order), or in even higher order dynamics. To fully grasp the functions of a single node in the network, it is crucial to understand of its activities relative to the richly woven network that supports it. Therefore, incorporating information from higher order dynamics could potentially improve the quality of brain analysis.

\end{enumerate}

First, to find an effective replacement for the widely used sliding window method, we designed TimeCorr, an intuitive method that found inspiration directly from the fundamental correlation function. Like the sliding window method, TimeCorr is able is able to recover functional connectivity from fMRI brain images with similar if not higher accuracy. But TimeCorr goes beyond the sliding window method by (a) not requiring buffer at each end for calculation, thereby avoiding data loss, (b) offering extra stability by using all the time points in the time series for calculation of correlation at every time point, (c) provides the option to shift between locality and fluidity based on user demand by varying the resolution parameter.\\

TimeCorr achieves all the above functions through the use of Gaussian distribution. By attaching a Gaussian coefficient to each time point component in the time series, TimeCorr is able to allocate the amount of influence each neighboring time point has on the calculation of correlation at the time point of interest. As the Gaussian center (highest density/coefficient) is always at the time point of interest, TimeCorr guarantees to return results that are highly proportional to the local ground truth. In addition, the user can choose to have more fluidity in overall results by choosing coefficients from a Gaussian distribution with a large variance; or higher resolution and more locality by choosing coefficients from a distribution with a smaller variance.\\

Second, we designed a "level up" method that applies TimeCorr, Inter-Subject Functional Connectivity (ISFC) and Principle Component Analysis (PCA) to calculate higher order functional connectivity. To overcome the problem of scaling and to enforce uniformity for inter-levels analysis, we use TimeCorr on each subject to calculate functional connectivity from previous level activations and then reduce the results to an arbitrary number of feature points to represent the activations for the next level. As TimeCorr avoids data loss in calculating functional connectivity and PCA maintains an uniform number of activations at each "level up", we are able to obtain $10^{th}$ order functional connectivity while ensuring linear scaling in storage and computation.\\

We believe that different orders of functional connectivity represents information on different aspects of brain dynamics, and incorporating knowledge of the higher-order structures of the brain will expand present brain decoding capabilities. To measure the amount of information present at each level, we introduce decoding accuracy, a testing parameter that describes the proportion of time points in one level of a subject's functional connectivity graph that has the highest correlation with the same time point from the average activations of all other subjects. We are interested in how decoding accuracy changes as we reach higher order functional connectivity. Although we presumed brain dynamics would converge as we get to higher orders and decoding accuracy would increase, reality proved otherwise. This is shown and discussed in more detail in the Intra-Level Decoding Analysis section of our results.\\

Finally, we present the High-Order Brain Dynamics (HOBD) model, a method that seeks to bring everything together. By finding an optimal mixture weight for each level, the HOBD is able to efficiently incorporate information from multiples orders of functional connectivity obtained using the "level up" approach above and achieve significantly higher decoding accuracy than any level alone. In the Multi-Level Mixture Analysis section of our results, we provide a side by side comparison to highlight the improvements HOBD achieved in its decoding capabilities.\\



\begin{enumerate}
\item Advantages of using the TimeCorr method
\item The goal and focus of this project
\begin{enumerate}
\item Develop a general toolbox to help related research in inter-subject functional connectivity research
\item Conduct analysis on datasets to understand the significance of higher levels in decoding datasets (Mixture Analysis)
\end{enumerate}
\end{enumerate}


\subsection{Single Subject TimeCorr}
Find method to find dynamic correlation without loss of data

Effectively utilize information from entire dataset to achieve higher accuracy

Inspiration from the regular correlation function, but applies normalized Gaussian density function so that each time point influences the calculation of dynamic correlation at time point of interest proportional to its distance from the time point of interest.

Given a single subject fRMI dataset with T time points and V voxels, the TimeCorr method is used to determine the temporal correlation between voxels at each time points. \\
\begin{enumerate}
\item Choose variance V to represent the amount of influence neighboring time points will have on the calculation of the voxel correlation at each time point.\\
\item For each time point t:
\begin{enumerate}
\item Generate an array of Gaussian coefficients $w_t$ of length T and variance V with center at the $t^{th}$ element\\
\item Element-wise multiply the time series of activations of every voxel $a_i$ with the Gaussian coefficients array to create weighted activations array for each voxel $S^i_t$\\
\item Find the temporal dynamic correlation between voxel i and voxel j at time point t through the equation:
\begin{align*}
C(S^i_t,S^j_t) = \frac{1}{Z}\frac{\sum_{l=0}^T (S_l^i - \bar{S^i_t})\cdot(S^j_l - \bar{S^j_t})\cdot \mathcal{N}(l|t,\sigma)}{\sigma_{S_t^i} \cdot \sigma_{S_t^j}}
\end{align*}
Where
\begin{align*}
Z &= \sum_{l=0}^T \mathcal{N}(l|t,\sigma)\\
\bar{S^i_t} &=\frac{1}{Z} \sum_{l=0}^T S^i_l \cdot \mathcal{N}(l|t,\sigma)\\
\sigma_{S_t^i} &=\sqrt{ \frac{1}{Z}\sum_{l=0}^T (S_l^i-\bar{S_t^i})^2 \cdot \mathcal{N}(l|t,\sigma)}\\
\end{align*}
\item Repeat above process for every voxel pair to create correlation matrix for time point t
\end{enumerate}
\end{enumerate}

After experimenting with different setups, we discovered that the best way to distribute weights for time points in the time series is to apply a Gaussian probability density function centered around the time point of interest with variance equal to the total number of time points. This finding will be discussed in more detail in the Results section.\\

In contrast to the sliding window approach which is widely considered as the golden standard for fRMI dynamic correlation calculation, the TimeCorr approach is able to more accurately retrieve the temporal correlation at each time point without loss of important data due to limitations from using time frames. This advantage allows us to "level up" the voxel activation matrix many times and calculate correlation at higher orders while maintaining information at all time points.\\

\subsection{Inter-Subject Connectivity using TimeCorr}
The ISFC is a proces through which we find the stimulus-dependent activations in our fMRI dataset by cross referencing and averaging data from multiple subjects.\\
\begin{enumerate}
\item For each subject s, we find the average activation of all other subjects:
\begin{align*}
O_s=\frac{\sum_{i\neq s}^N S_i}{N-1}
\end{align*}
where $S_i$ represents the activation matrix for subject $i$ and $N$ represents the total number of subjects.
\item Find the correlation matrices between the voxel activations for each subject $S_i$ and the average voxel activations of all other subjects $O_{i}$ using the TimeCorr method with variance $\sigma$. To find the correlation between voxel activation $S^i_t$ of subject $S$ for voxel $i$ at time $t$ and voxel activation $O^j_t$ of the average of other subjects for voxel $j$ at time points $t$ is obtained through the following equation:
\begin{align*}
C(S^i_t,O^j_t) = \frac{1}{Z}\frac{\sum_{l=0}^T (S_l^i - \bar{S^i_t})\cdot(O^j_l - \bar{O^j_t})\cdot \mathcal{N}(l|t,\sigma)}{\sigma_{S_t^i} \cdot \sigma_{O_t^j}}
\end{align*}
Where
\begin{align*}
Z &= \sum_{l=0}^T \mathcal{N}(l|t,\sigma)\\
\bar{S^i_t} &=\frac{1}{Z} \sum_{l=0}^T S^i_l \cdot \mathcal{N}(l|t,\sigma)\\
\bar{O^i_t} &=\frac{1}{Z} \sum_{l=0}^T O^i_l \cdot \mathcal{N}(l|t,\sigma)\\
\sigma_{S_t^i} &=\sqrt{ \frac{1}{Z}\sum_{l=0}^T (S_l^i-\bar{S_t^i})^2 \cdot \mathcal{N}(l|t,\sigma)}\\
\sigma_{O_t^i} &=\sqrt{ \frac{1}{Z}\sum_{l=0}^T (O_l^i-\bar{O_t^i})^2 \cdot \mathcal{N}(l|t,\sigma)}\\
\end{align*}
\item Apply Fisher Z-transformation to every element $r$ of the correlation matrices for each subject at each time points to obtain the corresponding Z-correlation matrices:
\begin{align*}
z = \frac{1}{2}\ln(\frac{1+r}{1-r})
\end{align*}
\item Average the Z-correlation matrices $Z_i$ acros all subjects:
\begin{align*}
S_Z = \frac{1}{N}\sum^N_{i=1}Z_i
\end{align*}
\item Apply inverse Z-transformation to the average Z-correlation matrix to obtain the Inter-subject Functional Connectivity (ISFC) mean correlation matrix:
\begin{align*}
ISFC = \frac{\exp(S_Z+S_Z^T)-1}{\exp(S_Z+S_Z^T)+1}
\end{align*}
\end{enumerate}

\subsection{Multi-Subject TimeCorr Level-Up}
Add justification for using PCA\\

The multi-subject TimeCorr Level-Up process utilizes the functionalities of the single-subject TimeCorr method to extract high-order brain dynamics patterns from brain activation data. Given a brain activation or activation correlation matrix at level $l$ of dimensions S subject, V voxels and T time points, the Level-Up function\\
\begin{enumerate}
\item Applies single-subject TimeCorr on the data matrix for each subject to obtain correlations matrix\\
\item Concatenate the correlations matrix for each subject from the previous step together along the voxels dimension into a single matrix of dimesions $SVxT$\\
\item Apply PCA on the concatenated matrix from the previous step to obtain a reduced representation of the correlation matrix\\
\item Separate the reduced correlation matrix from the previous step into data for each subject to obtain a 3-D matrix of size SxVxT
\item Repeat the above process on the output to level up again
\end{enumerate}
Intuitively, every time the multi-subject TimeCorr Level-Up function is applied on a dataset, whether it's an activations matrix or a correlation matrix, the dynamic correlations matrix of the dataset is obtained. Using this process, we are not only able to calculate the correlation of brain activations, but also the correlation of the correlation of brain activations, etc. \\
Another function of the multi-subject TimeCorr Level-Up function that makes it standout from traditional methods is its application of the single-subject TimeCorr method in calculating dynamic correlations. Traditional methods use the sliding-window approach and lose significant portion of the data with each level, and thus can only level up a limited number of iterations. In contrast, because the single-subject TimeCorr method is able to retain all information of the original dataset after each application, it theoretically has the capability to level up a dataset infinite times. This advantage allows us to explore higher-order dynamic patterns within the brain that was previously impossible to access.\\

\subsection{Results}
\begin{enumerate}
\item Synthetic dataset generation
\item Synthetic dataset testing
\begin{enumerate}
\item Testing single-subject TimeCorr on block correlation dataset and comparison with sliding window results
\item Testing single-subject TimeCorr on ramp correlation dataset and comparison with sliding window results
\item Testing ISFC on block correlation dataset and comparison with sliding window results
\item Testing ISFC on ramp correlation dataset and comparison with sliding window results
\item Testing level up on multisubject ramping correlation dataset and comparison with sliding window results
\end{enumerate}
\item Results on the Pieman dataset
\begin{enumerate}
\item See Jeremy's paper for reference on order and structure of this section\\
\item Pieman Intact
\end{enumerate}
\item Results on Sherlock dataset
\item Results on Forrest dataset
\item Level-mixture analysis
\item Optimization and Benchmark results

\end{enumerate}
\subsection{Conclusion}
\begin{enumerate}
\item application in medicine for classification data enrichment
\end{enumerate}
\begin{enumerate}
\item \cite{jeremy2017} decoding analysis, ISFC
\item \cite{hasson2016} First time Inter-subject functional connectivity is introduced. opens new avenues for linking brain network dynamics to stimulus features and behavior.
\item \cite{khambhati2017} We describe recent efforts to model dynamic patterns of connectivity, dynamic patterns of activity, and patterns of activity atop connectivity. In the context of these models, we review important considerations in statistical testing, including parametric and non-parametric approaches. Finally, we offer thoughts on careful and accurate interpretation of dynamic graph architecture, and outline important future directions for method development.
\item \cite{davidson2016} we present a method to quantify individual differences in brain functional dynamics by applying hypergraph analysis, a method from dynamic network theory. age-related changes in brain function can be better understood by taking an integrative approach that incorporates information about the dynamics of functional interactions
\item \cite{peterson11} However, we find that observations of “dynamic” BOLD correlations during the resting state are largely explained by sampling variability. Beyond sampling variability, the largest part of observed “dynamics” during rest is attributable to head motion. An additional component of dynamic variability during rest is attributable to fluctuating sleep state. Thus, aside from the preceding explanatory factors, a single correlation structure—as opposed to a sequence of distinct correlation structures—may adequately describe the resting state as measured by BOLD fMRI. These results suggest that resting-state BOLD correlations do not primarily reflect moment-to-moment changes in cognitive content. Rather, resting-state BOLD correlations may predominantly reflect processes concerned with the maintenance of the long-term stability of the brain’s functional organization
\item \cite{peterson12} Machine Learning for classification. We also report a novel adaptation of SVM binary classification that, in addition to an overall accuracy rate for the SVM, provides a confidence measure for the accurate classification of each individual. Our results support the contention that multivariate methods can better capture the complexity of some brain disorders, and hold promise for predicting prognosis and treatment outcome for individuals with TS.
\item \cite{peterson17} Support vector regression enabled quantitative estimation of birth gestational age in single subjects using only term equivalent resting state-functional MRI data, indicating that the present approach is sensitive to the degree of disruption of brain development associatedwith pretermbirth (using gestational age as a surrogate for the extent of disruption). This suggests that support vector regression may provide a means for predicting neurodevelopmental outcome in individual infants.
\item \cite{olaf2005} First time research is conducted in brain functional connectivity, it's significance is revealed.
\item \cite{hasson2012} Here we show that temporal patterns of neural activity contain information that can discriminate different stimuli, even within brain regions that show no net activation to that stimulus class. Furthermore, we find that in many brain regions, responses to natural stimuli are highly context dependent. In such cases, prototypical event-related responses do not even exist for individual stimuli, so that averaging responses to the same stimulus within different contexts may worsen the effective signal-to-noise. As a result, analysis of the temporal structures of single events can re
\item \cite{enrico2011} Sliding window, instantaneous phase synchronization to increase temporal resolution, Dynamic Functional Connectivity
\item \cite{Greene01} The long-standing rationalist tradition in moral psychology emphasizes the role of reason in moral judgment. A more recent trend places increased emphasis on emotion. Although both reason and emotion are likely to play important roles in moral judgment, relatively little is known about their neural correlates, the nature of their interaction, and the factors that modulate their respective behavioral influences in the context of moral judgment. In two functional magnetic resonance imaging (fMRI) studies using moral dilemmas as probes, we apply the methods of cognitive neuroscience to the study of moral judgment. We argue that moral dilemmas vary systematically in the extent to which they engage emotional processing and that these variations in emotional engagement influence moral judgment.
\item \cite{Logothetis01} Functional magnetic resonance imaging (fMRI) is widely used to study the operational organization of the human brain, but the exact relationship between the measured fMRI signal and the underlying neural activity is unclear. These findings suggest that the BOLD contrast mechanism reflects the input and intracortical processing of a given area rather than its spiking output.
\item \cite{hasson2005} AUDIO The predicted fMRI signals derived from single units and the measured fMRI signals from auditory cortex showed a highly significant correlation (r 0 0.75, P G 10j47). Thus, fMRI signals can provide a reliable measure of the firing rate of human cortical neurons.
\item \cite{hasson2004} VIDEO The results reveal a surprising tendency of individual brains to “tick collectively” during natural vision. The intersubject synchronization consisted of a widespread cortical activation pattern correlated with emotionally arousing scenes and regionally selective components. The characteristics of these activations were revealed with the use of an open-ended “reverse-correlation” approach, which inverts the conventional analysis by letting the brain signals themselves “pick up” the optimal stimuli for each specialized cortical area.
\item \cite{peterson9} various denoising methods
\item \cite{peterson19} We suggest that, in the place of a single localized error mechanism, these findings point to a large-scale set of error-related regions across multiple systems that likely subserve different functions.
\item \cite{peterson20} In 2011, three groups reported that small headmovements produced spurious but structured noise in brain scans, causing distance-dependent changes in signal correlations. This finding has prompted both methods development and the re-examination of prior findings with more stringentmotion correction
\end{enumerate}
\begin{thebibliography}{1}
\bibitem{jeremy2017} Jeremy Manning, Xia Zhu, Theodore Willke, Rajesh Ranganath, Kimberly Stachenfeld, Uri Hasson, David M Blei, Kenneth A Norman. A probabilistic approach to discovering dynamic full-brain functional connectivity patterns. \textit{bioRxiv} 106690, 2017
\bibitem{elena2012} Elena A. Allen, Eswar Damaraju, Sergey M. Plis, Erik B. Erhardt, Tom Eichele and Vince D. Calhoun. Tracking Whole-Brain Connectivity Dynamics in the Resting State. \textit{Cerebral Cortex} March 2014;24:663–676
\bibitem{enrico2011} Enrico Glerean, Juha Salmi, Juha M. Lahnakoski, Iiro P. Jaaskelainen, and Mikko Sams. Functional Magnetic Resonance Imaging Phase Synchronization as a Measure of Dynamic Functional Connectivity. \textit{BRAIN CONNECTIVITY} Volume 2, Number 2, 2012
\bibitem{olaf2005} Olaf Sporns, Giulio Tononi, Rolf Kotter. The Human Connectome: A Structural Description of the Human Brain. \textit{PLOS Computational Biology} September 2005
\bibitem{hasson2016} Erez Simony, Christopher J. Honey, Janice Chen, Olga Lositsky, Yaara Yeshurun, Ami Wiesel and Uri Hasson. Dynamic reconfiguration of the default mode network during narrative comprehension. \textit{nature communications} July, 2016
\bibitem{davidson2016} Elizabeth N. Davison, Benjamin O. Turner, Kimberly J. Schlesinger, Michael B. Miller, Scott T. Grafton, Danielle S. Bassett, Jean M. Carlson. Individual Differences in Dynamic Functional Brain Connectivity Across the Human Lifespan. \textit{arXiv}:1606.09545v1
\bibitem{tang2017} Evelyn Tang and Danielle S. Bassett. Control of Dynamics in Brain Networks. \textit{arXiv}:1701.01531v2
\bibitem{khambhati2017} Ankit N. Khambhati, Ann E. Sizemore, Richard F. Betzel, and Danielle S. Bassett. Modelling and Interpreting Network Dynamics. \textit{bioRxiv} Apr. 4, 2017.
\bibitem{hasson2009} Uri Hasson, Rafael Malach and David J. Heeger. Reliability of cortical activity during natural stimulation. \textit{Cell Press} December 2009
\bibitem{hasson2005} Roy Mukamel, Hagar Gelbard, Amos Arieli, Uri Hasson, Itzhak Fried and Rafael Malach. Coupling Between Neuronal Firing, Field Potentials, and fMRI in Human Auditory Cortex. \textit{SCIENCE} VOL 309 5 AUGUST 2005
\bibitem{hasson2004} Uri Hasson, Yuval Nir, Ifat Levy, Galit Fuhrmann and Rafael Malach. Intersubject Synchronization of Cortical Activity During Natural Vision. \textit{SCIENCE} 12 MARCH 2004 VOL 303.
\bibitem{hasson2012} Aya Ben-Yakova, Christopher J. Honey, Yulia Lerner and Uri Hasson. Loss of reliable temporal structure in event-related averaging of naturalistic stimuli. \textit{NeuroImage} 63 (2012) 501–506
\bibitem{zubeidi2014} Duha Al-Zubeidi, Mathula Thangarajh, PhDb, Sheel Pathak, Chunyu Cai, Bradley L. Schlaggar, Gregory A. Storch, Dorothy K. Grange and Michael E. Watson Jr.
\bibitem{peterson9} Gregory C. Burgess, Sridhar Kandala, Dan Nolan, Timothy O. Laumann, Jonathan D. Power, Babatunde Adeyemo, Michael P. Harms, Steven E. Petersen, and Deanna M. Barch. Evaluation of Denoising Strategies to Address Motion-Correlated Artifacts in Resting-State Functional Magnetic Resonance Imaging Data from the Human Connectome Project. \textit{BRAIN CONNECTIVITY} Volume 6, Number 9, 2016
\bibitem{peterson10} Evan M. Gordon, Timothy O. Laumann, Babatunde Adeyemo, Adrian W. Gilmore, Steven M. Nelson, Nico U.F. Dosenbach, Steven E. Petersen. Individual-specific features of brain systems identified with resting state functional correlations. \textit{NeuroImage} 146 (2017) 918–939
\bibitem{peterson11} Timothy O. Laumann, Abraham Z. Snyder, Anish Mitra, Evan M. Gordon, Caterina Gratton, Babatunde Adeyemo, Adrian W. Gilmore, Steven M. Nelson, Jeff J. Berg5, Deanna J. Greene, John E. McCarthy, Enzo Tagliazucchi, Helmut Laufs, Bradley L. Schlaggar, Nico U. F. Dosenbach, and Steven E. Petersen. On the Stability of BOLD fMRI Correlations
\bibitem{peterson12} Deanna J. Greene, Jessica A. Church, Nico U.F. Dosenbach, Ashley N. Nielsen, Babatunde Adeyemo, Binyam Nardos, Steven E. Petersen, Kevin J. Black and Bradley L. Schlaggar. Multivariate pattern classification of pediatric Tourette syndrome using functional connectivity MRI. \textit{Developmental Science} 19:4 (2016), pp 581–598
\bibitem{peterson17} Christopher D. Smyser, Nico U.F. Dosenbach, TaraA. Smyser, AbrahamZ. Snyder, Cynthia E. Rogers, Terrie E. Inder, Bradley L. Schlaggar and Jeffrey J. Neil. Prediction of brain maturity in infants using machine-learning algorithms. \textit{NeuroImage} 136 (2016) 1–9
\bibitem{peterson19} Maital Neta, XFrancis M. Miezin, Steven M. Nelson, Joseph W. Dubis, Nico U.F. Dosenbach, Bradley L. Schlaggar and Steven E. Petersen. Spatial and Temporal Characteristics of Error-Related Activity in the Human Brain. The Journal of Neuroscience, January 7, 2015
\bibitem{peterson20} Jonathan D. Power, Bradley L. Schlaggar, and Steven E. Petersen. Recent progress and outstanding issues in motion correction in resting state fMRI. \textit{NeuroImage} 105 (2015) 536–551
\bibitem{Logothetis01} Nikos K. Logothetis, Jon Pauls, Mark Augath, Torsten Trinath and Axel Oeltermann. Neurophysiological investigation of the basis of the fMRI signal. \textit{Nature} 412, 150-157 (12 July 2001)
\bibitem{Greene01}Joshua D. Greene, R. Brian Sommerville, Leigh E. Nystrom, John M. Darley and Jonathan D. Cohen. An fMRI Investigation of Emotional Engagement in Moral Judgment. \textit{Science}  14 Sep 2001: Vol. 293, Issue 5537
\bibitem{Friston99} K.J. Friston, A. Holmes, C.J. Price, C. Buchela and K.J. Worsley. Multisubject fMRI Studies and Conjunction Analyses. \textit{NeuroImage} Volume 10, Issue 4, October 1999
\bibitem{Friston98} K.J. Friston, P. Fletcher, O. Josephs, A. Holmes, M.D. Rugg and R. Turner. Event-Related fMRI: Characterizing Differential Responses. \textit{NeuroImage} Volume 7, Issue 1, January 1998
\bibitem{Ogawa90} S. Ogawa, T. M. Lee, A. R. Kay, and D. W. Tank. Brain magnetic resonance imaging with contrast dependent on blood oxygenation. \textit{PNAS} December 1, 1990
\bibitem{Norman06} Kenneth A. Norman, Sean M. Polyn, Greg J. Detre and James V. Haxby. Beyond mind-reading: multi-voxel pattern analysis of fMRI data. \textit{Trends in Cognitive Science} (2006)
\bibitem{Turke13} Nicholas B. Turk-Browne. Functional Interactions as Big Data in the Human Brain. \textit{SCIENCE} (342) November 1st, 2013.
\bibitem{Rubinov2010} M. Rubinov and O. Sporns. Complex network measures of brain connectivity: uses and interpretations. \textit{NeuroImage}, 52:1059-1069, 2010.
\bibitem{Betzel2017} R. E. Betzel, J. D. Medaglia, L. Papadopoulos, G. Baum, R. Gur, R. Gur, D. Roalf, T. D. Satterthwaite, and D. S. Bassett. The modular organization of human anatomical brain networks: accounting for the cost of wiring. \textit{Network Neuroscience}, page Advance online publication. doi:10.1162/netn.a.00002., 2017.
\bibitem{Craddock2012} R. C. Craddock, G. A. James, I. P E Holtzheimer, X. P. Hu, and H. S. Mayberg. A while brain fmri atlas generated via spatially constrained spectral clustering. \textit{Human Brain Mapping}, 33(8):1914-1928, 2012.
\bibitem{Yeo2011} B. T. T. Yeo, F. M. Krienen, J. Sepulcre, M. R. Sabuncu, D. Lashkari, M. Hollinshead, J. L. Roffman, J. W. Smoller, L. Zollei, J. R. Polimieni, B. Fischl, H. Liu, and R. L. Buckner. The organization of the human cerebral cortex estimated by intrinsic functional connectivity. \textit{Journal of Neurophsiology}, 106(3):1125-1165, 2011.
\end{thebibliography}
\end{document}
